\chapter{Standards and Regulations}
\label{cha:standards}

Until the so-called third industrial age, or Industry 3.0, cybersecurity had minimal impact on manufacturing. Industrial machinery wasn't necessarily connected to the internet or each other, making external risks unlikely. However, with the dawn of Industry 4.0, smart machinery and smart factories have become vital for the smooth operation of production departments, placing cybersecurity at the forefront of concern.

Cybersecurity involves protecting systems, networks, and programs from digital attacks. Given the critical nature of the topic and the attention it demands, companies are embarking on paths to elevate their awareness of cyberattacks, adopting internal policies, or even pursuing specific certifications in cybersecurity. Concurrently, national and international legislators have introduced new legislative measures imposing new obligations on certain entities in relation to cybersecurity.

Let's now consider the most recent legislative measures on cybersecurity and the related certifications that manufacturing industries must take into account.\\
Certifications provide a guarantee of product and company security. The main certifications currently considered include:
\begin{itemize}
  \item \textbf{IEC 62443}: This series of international standards is renowned for enhancing the security of industrial control systems, setting forth fundamental prerequisites to shield industrial systems from cyber threats;
  \item \textbf{ISO/IEC 27001 (2022)}: This certification covers various aspects, including security policy, human resource security, physical and environmental security, communications management, and regulatory compliance;
\end{itemize}

There are then several laws and regulations that address under various profiles the issue of cybersecurity among them it is worth noting:~\cite{cybersecurity-standards-regulations-compliance}
\begin{itemize}
  \item \textbf{Cyber Resilient Act (CRA)}\footnote{\url{https://eur-lex.europa.eu/legal-content/EN/TXT/?uri=celex:52022PC0454}}: While still pending final approval by the European Institutions, the Cyber Resilient Act seeks to enhance the resilience of the European digital market by ensuring that connected devices and digital services are equipped to withstand cyberattacks effectively. It will become effective in 2027;
  \item  \textbf{NIS Directive 2}\footnote{\url{https://eur-lex.europa.eu/legal-content/EN/TXT/?uri=CELEX:32022L2555}}: This directive outlines essential criteria that companies must adhere to in order to maintain a robust level of cybersecurity. These criteria encompass the implementation of risk analysis strategies, fortification of information system security, and effective incident management protocols. It is imperative for EU Member States to transpose this directive by October 2024, with enforcement timelines specified in the respective national transposition acts;
  \item \textbf{Italian Cybersecurity Law (June 28th, 2024 No. 90)}\footnote{\url{https://www.gazzettaufficiale.it/eli/id/2024/07/02/24G00108/sg}}: This Italian law pertains to national cybersecurity and applies to both public and private entities whose services are deemed critical. It mandates the implementation of security measures to protect critical digital infrastructures and sensitive information, including specific obligations to notify the Cybersecurity Agency of any cyber incidents;
  \item \textbf{New Machinery Regulation No. 1230/2023}\footnote{\url{https://eur-lex.europa.eu/legal-content/EN/TXT/?uri=CELEX:32023R1230}}: This regulation will replace the Machinery Directive No. 2006/42/EC, focusing on the overall safety of machinery and semi-machinery and it will become effective in 2027. It emphasizes the essential integration of cybersecurity into the design and manufacturing processes of machinery, recognizing the potential risks that cyber vulnerabilities pose to physical safety;
\end{itemize}

This chapter will provide an overview of the most relevant standards and regulations for the context of the internship's researches, focusing on the industrial sector and the required goals; in particular, we will detail more about \textit{IEC 62443}, \textit{ISO 27001} and \textit{CRA}.

\section{IEC 62443}

The IEC 62443 provides guidelines, rules and definitions specifically crafter for any Industrial Control System (ICS) and Industrial Automation and Control Systems (IACS). Compared to ISO 27001, it is more focused on the specific sector instead of being more universal and open for the interpretation depending on the company it applies on.

The market is starting to require the IEC 62443 certification for the companies that are involved in the production of industrial devices, as it is a guarantee of the security of the product and the company itself. 

The IEC 62443 is a set of standards drafted by the \textit{Internation Electrotechnical Commission} (IEC) and it is divided into four parts:~\cite{understanding-iec-62443-parts}
\begin{enumerate}
  \item General: it covers topics that are common to the entire series;
  \item Policies and procedures: focuses on methods and processes associated with IACS security;
  \item System: it covers the requirements for the secure development and integration of systems;
  \item Component: it covers the requirements for the secure development and integration of components.
\end{enumerate}






\section{ISO/IEC 27001}






