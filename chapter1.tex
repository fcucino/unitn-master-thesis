\chapter{Introduction}
\label{cha:intro}

Nowadays, cybersecurity is one of the most trending subjects of discussion around the world~\cite{trends-computer-science}.

The \textit{Internet} is growing: the users are growing, therefore the exchanged amount of data are growing, and so the malicious actors are growing, trying to mislead an increasing number of people, regardless they are newbies or experienced ones. Every day hundreds of new vulnerabilities~\cite{cve-details-db} are discovered and millions of cyberattacks are performed, with an increase of nearly three times compared to the previous year~\cite{cybersecurity-attacks}.

In today's rapidly evolving industrial landscape, the integration of connected devices and smart technologies, commonly referred to as the Internet of Things (IoT), has transformed traditional industries such as manufacturing, energy and transportation. These connected devices enable real-time monitoring, predictive maintenance and automation, resulting in enhanced productivity and efficiency. \\
However, the widespread adoption of the Industrial IoT also introduces significant security challenges: industrial control systems and operational technology (OT) networks, which were historically isolated, are now exposed to cyber threats that were once only confined to the realm of information technology (IT). \\
The increased interconnectivity has expanded the attack surface, making industrial devices vulnerable to cyberattacks, including ransomware, data breaches and operational disruptions. According to the European Commission, hardware and software products are increasingly subject to successful cyberattacks, leading to an estimated global annual cost of cybercrime of EUR 5.5 trillion by 2021~\cite{cra-eu}.

The current situation of many industrial environments is inadequate to meet the rising threat level. Many legacy systems and industrial devices lack modern security features and traditional security measures are often difficult to implement due to the unique requirements of industrial environments. To address these challenges, there is a pressing need for security solutions tailored to the specific characteristics of industrial devices and networks.

This thesis presents the design and the implementation of an application that performs security scans on industrial devices. The objective is to develop a robust and efficient tool capable of identifying potential risks due to misconfigurations in industrial devices, ensuring the integrity, confidentiality and availability of the device and its associated data and providing actionable recommendations for improving the security of industrial systems. We suggest the best practices and do not directly modify the devices configurations due to the critical nature of industrial systems, where any disruption could have severe consequences on the physical environment.

The internship project and this thesis are part of the curricular activities of the Master's Degree in Computer Science - Cybersecurity branch - at the University of Trento. The project was carried out in the first semester of 2024 in a company called \textit{Corvina}\footnote{\url{https://corvina.io}}, part of the \textit{Exor International}\footnote{\url{https://www.exorint.com}} multinational group, located in the province of Verona, which designs and builds industrial devices and provides solutions to make these legacy devices \textit{smart} by connecting them to the Internet and exchanging data with the Corvina cloud platform, in order to let the customers retrieve and elaborate these data or remote controlling the devices.

In the following chapters, the underlying design principles, implementation strategies and practical applications of the proposed security scan tool will be discussed in detail, providing a comprehensive solution to an increasingly critical problem in industrial cybersecurity.
